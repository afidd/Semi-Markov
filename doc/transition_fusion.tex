%\documentclass[review]{elsarticle}
\documentclass{article}
\usepackage{hyperref}
\usepackage{setspace}
%\usepackage[margin=.5in]{geometry}
\usepackage{xspace}
\usepackage{xcolor}
%\usepackage{amsmath}
\usepackage{graphicx}
%%\doublespacing

\begin{document}

\title{Transition Fusion for Semi-Markov Library}

\author{Drew Dolgert}
\date{21 August 2014}

\section{Sets To Define Transitions}
Transition fusion is a way to merge GSPN by merging transitions,
not places, as is usually done. The two processes are equivalent,
but merging fusions can be clearer for analysis.

The literature on transition fusion is for Petri nets, for which
a transition is enabled when tokens are on its input places and
fires by moving tokens to its output places. In this context,
each transition is given an identifier, which we'll call a key
in a space T. Two graphs fuse by a rule $(T,T)→T$. This rule
creates a \emph{new} transition. The rule, alone, is sufficient
to define the enabling and firing of a transition in the context
of a Petri net.

A GSPN Transition has more properties than a Petri net transition.
The enabling rule is an arbitrary function on its input places,
as is the firing rule. Each transition also has a continuous
distribution determined by a funciton on its input and output places.

This document is about applying transition fusion to a particular
case. A contact graph is given, where every node contains one
of a set of GSPN. The goal is to construct a single GSPN from
the nodes and a set of rules which create nodes associated with
each edge, connecting the GSPN at the nodes.

For the Semi-Markov library, every place and transition has a
unique, assigned name, called a key. The C++ graph implementation
unfortunately insists on associating each vertex in the graph
with an opaque vertex descriptor of its own choosing, as well.
Stored separately is a map from vertex descriptor to key and
vertex descriptor to a transition object or place object.

This fusion tasks doesn't start with separate graphs in the
same space, $T$. It starts with a set of graphs in their
own spaces, say $A$ and $B$. The contact graph defines
a space of nodes, $N$, as well. Let's say $A$ is SIR
on a dairy farm, so the transition keys represent $(herd, disease state)$.
Then $B$ is a slaughtering facility, so the keys are
$(processing step, disease state)$. There is no guarantee
that the tuples are in the same space or, if they are, that
they do not overlap.

The simplest space would be a superset of the given ones,
$C=(A,A,B,B,N,N)$. Since transitions will be associated with edges,
there are two $N$ entries and two of each type of the node.
This seems like too much. Let's return to the rules. For this
construction, they are of the form,
$(A,A)\rightarrrow D$, $(B,B)\rightarrrow D$, $(A,B)\rightarrrow D$,
or $(B,A)\rightarrrow D$. We could define $A→D$ and $B→D$ so
that the unified space is $C=(D,N,N)$.

It feels like any transition, identified first by its vertex
descriptor, has an identity that can be traced through a series
of maps, $C$ to $D$, $D$ to $A$, $A$ to disease transition.
This tracing can happen at every time step because an observer will
track certain changes to the state of the system.

How many types of transitions are in the system?
$\#A+\#B+\#(rules)$. The space for $D$ would be small, so maybe
this could be built as an explicit map, except for the darned
disparate types.

Returning to the fusion of transitions, one possible sequence is
the following.

\begin{enumerate}
\item Create a complete set of places and transitions just by promoting
each node in the contact graph to $C$. This graph will now have
virtual transitions, which means those ready to be fused. Every 
transition is now in the $C$ space.

\item Iterate the original contact graph, armed with the rules
and instances of $A$ and $B$, to create a new set of rules
from $(C,C) → C$.

\item Return with the new set of rules to the larger graph and
create the new transitions.

\item Remove all virtual transitions.
\end{enumerate}

For a Petri net, fusing transitions means creating a new one
with specified inputs and outputs. The new transition in a GSPN
isn't simply a fusion because it has three parts, enabling, firing,
and distribution. Therefore this creation of new transition
from two previous ones would be a general function. There could
be a default rule, useful in some cases, which combines inputs
and outputs, uses a new enabling rule which checks whether either
was enabled, and fires both firing rules when firing the fused
transition. The continuous distribution, however, seems logically
up for grabs. Maybe using the first transition's distribution
rule could work.


\section{An Idea to Use Places more, Transition Keys less}
Because this is transition fusion, the places are all untouched
in the sense that they retain their original keys and will just
get put into other sets. Maybe what's needed is to forgo transition
keys as unique and instead just have some transition identifier,
intended to distinguish types of transitions, and then make it
easy in the code to ask what are the places associated with a transition.
This has to be efficient anyway.

So the places are a hierarchy of identifiers. The transitions have
only one identifier, sufficient to distinguish two transitions which
have the same set of input and output places.

This strategy might fail for colored transitions, for which
you would want to see the transition instance, itself, because
the color of the token is stored in the instance.

\section{Space in Which to Do Fusion}
25 August 2014.
The code now allows user transition keys which aren't unique.
Computation uses vertex\_descriptors as the ``keys'' for identification
of transitions. The reason to have any kind of user key for transitions
and places is so that observer code can figure out what it is that
happened. Non-unique transitions can be fine as long as we can identify
places attached to the transitions, so there is now a method on the
ExplicitTransitions class to make it simpler to ask what places associate
with a particular transition. Given that this query is exactly what
the main algorithm needs for speed, it is $\mathcal{O}(1)$ anyway, so it's
probably a better choice than having unique transitions. Any user transition
key attached to a transition would now be used to distinguish multiple
transitions with the same input and output places.

What form will the transition algorithm take? Specifically, will it use
vertex\_descriptors or user keys? Specification will include two
types of graphs, the contact graph and the subgraphs. Subgraphs
have vertex\_descriptors which are under control of the Boost Graph
Library and will likely not be unique between subgraphs. Different
types of subgraphs, which we will call $A$ and $B$ in examples, may
not have unique user keys, either, but we could enforce them to be unique.

The contact graph will need to specify an undirected graph with some
identifier for which subgraph is at which node. It has to be any
unique identifier which can then be mapped to a list of subgraph types.
For example, ``dairy farm'' versus ``processing plant.'' In addition, the
contact graph nodes will likely have other node properties, for example
latititude and longitude or names of some sort. For our purposes,
the contact graph nodes each have a pair, (unique node name, subgraph type).
The unique node name can be either a vertex\_descriptor or a user-supplied
unique name.

What is a fusion? It would be specified as a pair of transitions.
Remember the user transition keys may not be unique. It could be
(subgraph type, user transition key), (subgraph type, user transition key).
It could be (subgraph type, vertex\_descriptor),
(subgraph type, vertex\_descriptor). I see a sequence of specifications

\begin{enumerate}
  \item pair of (subgraph type, user transition key)
  \item pair of (subgraph type, vertex within subgraph)
  \item pair of (vertex within larger graph)
\end{enumerate}

When places are inserted into the larger graph, they are given user
transition keys that are of the form (unique node name, subgraph user place key).
When transitions are inserted into the larger graph, they are given the
name (subgraph user transition key). During the build process,
we need to be able to identify the subgraph vertex type, so we need to
track a map between (subgraph vertex\_descriptor) for a unique node name and
(larger graph vertex\_descriptor). This is a one-way map that exists
only while building the larger graph.
